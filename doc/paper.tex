\documentclass[onecolumn,10pt,cleanfoot]{asme2ej}

\usepackage{graphicx} %% for loading jpg figures
\usepackage{bm}
\usepackage{nicefrac}
\usepackage{mathtools}
\usepackage{amssymb}
\usepackage{amsmath}
\usepackage{parskip}
\usepackage{listings}
\usepackage{tablefootnote}
\usepackage{float}
\usepackage{xcolor}
\usepackage{xurl}

\author{Grzegorz Kajda
    \affiliation{
	Bachelor Student, Robotics and Intelligent Systems\\ \\[-10pt]
	Department of Informatics The faculty of Mathematics and Natural Sciences\\ \\[-10pt]
    Email: grzegork@ifi.uio.no
    }
}


\begin{document}

\title{Simulating the Lipkin-Meshov-Glikov model on a classical computer}

\maketitle

\begin{center}
\section{Abstract}

We present a comprehensive simulation of the simplified Hamiltonian of the Lipkin-Meshov-Glikov (LMG) model on a classical computer, employing the Variational Quantum Eigensolver (VQE) algorithm. Initially, we explore the outcomes arising from the application of diverse single qubit quantum gates to one qubit systems prepared in basis states. Subsequently, an entangler circuit is constructed, and one of the four bell pairs is prepared, to which we apply the Hadamard and CNOT gates. Utilizing our newfound grasp of the foundational principles governing quantum systems, we construct quantum circuits for generating ansatz states and methodically develop the VQE from ground up.

To assess the VQE's efficacy, we apply it to synthetic quantum systems representing single and two-particle systems, comparing its performance against numerical and analytical solvers. Transitioning to the Lipkin model, we leverage the properties of annihilation and creation operators to reformulate the Lipkin model's Hamiltonian first in terms of the quasi-spin operator and then Pauli operators. Employing this encoding scheme, we apply the VQE to systems with spin J=1 and 2, corresponding to Lipkin systems with two and four fermions, respectively. We compare the results against exact solutions obtained through standard diagonalization methods. This analysis provides valuable insights into the performance and accuracy of the VQE approach in tackling the Lipkin model's Hamiltonian.
	
\end{center}




\end{document}
