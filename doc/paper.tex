\documentclass[onecolumn,10pt,cleanfoot]{asme2ej}

\usepackage{graphicx} %% for loading jpg figures
\usepackage{bm}
\usepackage{nicefrac}
\usepackage{mathtools}
\usepackage{amssymb}
\usepackage{amsmath}
\usepackage{parskip}
\usepackage{listings}
\usepackage{tablefootnote}
\usepackage{float}
\usepackage{xcolor}
\usepackage{xurl}

\author{Grzegorz Kajda
    \affiliation{
	Bachelor Student, Robotics and \\
	Intelligent Systems\\ \\[-10pt]
	Department of Informatics\\ \\[-10pt]
	The faculty of Mathematics and \\
	Natural Sciences\\ \\[-10pt]
    Email: grzegork@ifi.uio.no
    }
}


\begin{document}

\title{Simulating the Lipkin-Meshov-Glikov model on a classical computer}

\maketitle

\begin{center}
\section{Abstract}

	We simulate the simplified Hamiltonian of the Lipkin-Meshov-Glikov model on a classical computer using the Variational Qunatum Eigensolver (VQE) algorithm. We start by investigating the outcomes resulting from the application of different single qubit quantum gates to qubit systems that are initially prepared in one of the basis states. Subsequently, we construct an entangler circuit and prepare one of the four bell pairs, upon which we apply the Hadamard and CNOT gates. With a fundamental comprehension of the foundational principles governing quantum systems, we contruct quantum circuits for generating ansatz states and meticulously develop the Variational Quantum Eigensolver (VQE) from ground up. We then apply the VQE to two synthetic quantum systems representing respectively one and two particle systems, and compare the performance of the VQE against numerical and analytical solvers. 
\end{center}



\end{document}
